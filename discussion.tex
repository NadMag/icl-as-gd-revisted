\subsection{Differences Between Theoretical Analysis and Proposed Methods}
Most works connecting ICL with gradient-based optimization are motivated by theoretical intuition or even include rigorous analysis for simplified settings \cite{pmlr-v202-von-oswald23a, akyürek2023learning} (see section~\ref{sec:related}).
Our work aims to demonstrate such connections empirically, guided by intuition provided in section \ref{sec:icl_dual_1}.  
It is important to note the differences between this analysis and practical settings: (1) It assumes linear attention is used (2) The analysis applies to the update of a single layer
(3) Starting from the ICL dual view, we get a partial-differential equation for the underlying loss function whose true value seems intractable.
While the empirical results provide direction for future research, we believe further analysis is needed in order to gain more insight from such comparisons. 

\subsection{Limitations and Future Directions}
The method and results shown in section \ref{sec:layer_causality} are inconclusive.
We address its limitation using quantitative analysis, which suggests that further modifications may yield better similarity with ICL.
We leave such work to future research.  