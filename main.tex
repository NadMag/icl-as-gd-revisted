% \pdfoutput=1

\documentclass[11pt]{article}

% \usepackage[review]{ACL2023}
\usepackage{ACL2023}

\usepackage{times}
\usepackage{latexsym}
\usepackage[T1]{fontenc}
\usepackage[utf8]{inputenc}
\usepackage{microtype}
\usepackage{inconsolata}
\usepackage{hyperref}
% \RequirePackage{algorithm}
% \RequirePackage{algorithmic}

\usepackage{multirow}
\usepackage{amsmath}
\usepackage{capt-of}
\usepackage{tabularx}
% \usepackage[caption=false]{subfig}
\usepackage{subcaption}
\usepackage{epsfig}
\usepackage{amssymb}
\usepackage{amsfonts}
\usepackage{booktabs}
\usepackage{scalerel}
%\usepackage[dvipsnames]{xcolor}
\usepackage[inline]{enumitem}
\usepackage{listings}
\usepackage{varwidth}
\usepackage[export]{adjustbox}
\usepackage{tikz}
\usetikzlibrary{tikzmark}
% \usepackage{todonotes}
% \usepackage{cleveref}
\newcommand{\crefrangeconjunction}{--}
\usepackage{stmaryrd}
\usepackage{bbm}
\usepackage{wrapfig}
\usepackage{pifont}

\newcommand{\tabincell}[2]{\begin{tabular}{@{}#1@{}}#2\end{tabular}}
\newcommand{\tx}[1]{``\textit{#1}''}
\newcommand{\sptk}[1]{\texttt{[#1]}}
% \newcommand{\eqform}[1]{Equation~(\ref{#1})}

% \usepackage{algorithm}
% \usepackage[noend]{algpseudocode}

\definecolor{deepblue}{rgb}{0,0,0.5}
\definecolor{officeblue}{RGB}{0,102,204}
\definecolor{deepred}{rgb}{0.6,0,0}
\definecolor{deepgreen}{rgb}{0,0.5,0}
\definecolor{mybrickred}{RGB}{182,50,28}
\newcommand\mybox[2][]{\tikz[overlay]\node[inner sep=1pt, anchor=text, rectangle, rounded corners=0mm,#1] {#2};\phantom{#2}}
\definecolor{fillcolor}{RGB}{216,217,252}
\newcommand\bg[1]{\mybox[fill=blue!20]{#1}}
\newcommand\rg[1]{\mybox[fill=red!20]{#1}}
\newcommand\graybox[1]{\mybox[fill=gray!20]{#1}}

% %%%%%Algorithm Box Package%%%%%
% \renewcommand{\algorithmicrequire}{\textbf{Input:}}
% \algnewcommand\algorithmicrequireb{{\hspace{0.85cm}}}
% \algnewcommand\INPTDESCB{\item[\algorithmicrequireb]}
% \renewcommand{\algorithmicensure}{\textbf{Output:}}
% \algnewcommand\algorithmicfuncdesc{\textbf{Function:}}
% \algnewcommand\FUNCDESC{\item[\algorithmicfuncdesc]}
% \algnewcommand\algorithmicfuncdescb{{\hspace{1.48cm}}}
% \algnewcommand\FUNCDESCB{\item[\algorithmicfuncdescb]}
% \algnewcommand{\algorithmicgoto}{\textbf{goto}}
% \algnewcommand{\Goto}[1]{\algorithmicgoto~\ref{#1}}
% \newcommand*\Let[2]{\State {#1 $\gets$ #2}}
% \newcommand*\LineLet[2]{#1 $\gets$ #2}
% %\newcommand*\AddOne[1]{\State #1 $\gets$ #1 $+1$}
% \newcommand*\AddOne[1]{\State #1 $++$}
% \newcommand*\LineComment[1]{\Statex \(\triangleright\) #1}
% \newcommand*\LineFor[2]{\State {\algorithmicfor~#1~\algorithmicdo~~~~#2}}
% \newcommand*\LineIf[2]{\State {\algorithmicif~#1~\algorithmicthen~~~~#2}}
% \newcommand*\AlgCommentInLine[1]{{\color{deepblue}{$\triangleright$ \textit{#1}}}}
% \newcommand*\AlgComment[1]{\State{\AlgCommentInLine{#1}}}
% %%%%%Algorithm Box Package%%%%%

\newcommand*\AlgCommentInLine[1]{{\color{deepblue}{$\triangleright$ \textit{#1}}}}

\input{math_commands.tex}

% \newcommand{\cmark}{\ding{51}\xspace}%
\newcommand{\cmarkg}{\textcolor{lightgray}{\ding{51}}\xspace}%
% \newcommand{\xmark}{\ding{55}\xspace}%
\newcommand{\xmarkg}{\textcolor{lightgray}{\ding{55}}\xspace}%

\newcommand\our{\textsc{structured prompting}}
\newcommand{\tblidx}[1]{{\scriptsize \texttt{[#1]}}}
\newtheorem{theorem}{Property}

\usepackage{pifont}% http://ctan.org/pkg/pifont
\newcommand{\cmark}{{\color{blue}\ding{51}}}%
\newcommand{\xmark}{{\color{black}\ding{55}}}%


\title{Rethinking In-Context Learning in Large Language Models as Gradient Descent}


\newcommand*\samethanks[1][\value{footnote}]{\footnotemark[#1]}

\author{\\
Tomer Bar Natan,~Gilad Deutch,~Nadav Magar\\
~~~~Supervised by Guy Dar\\
~~~~~Tel-Aviv University}

\date{}

\begin{document}

\maketitle

\begin{abstract}
	In-context learning (ICL) has shown impressive results in few-shot learning tasks, yet its underlying mechanism is still not fully understood.
	Recent works suggests that ICL could be thought of as an gradient descent (GD) based meta-optimization process.
	While promising, these results mainly focus on simplified settings of ICL and provide only preliminary evaluation of the similarities between the two methods. 
	In this work we revisit the comparison between ICL and GD based finetuning and study what properties of ICL an equivalent process must follow. 
	% Prediction level
	We begin in the model prediction level and reexamine how ICL and finetuning's prediction updates differ.
	%Motivated by these results we propose a novel evaluation metric which we term relative prediction alignment (RPA).
	% Layer Causality
	Next we address the differences in layer causality between ICL and standard finetuning.
	To study how this dissimilarity affects the model's behavior we propose a causally aware finetuning process and compare it with previous results. 
	%We find that the causally aware process competes with standard finetuning across several comparison metrics.
	% Linearalization
	Lastly we show that finetuning using a linear approximation of the model achieves comparable results to standard finetuning,
	suggesting that the underlying behavior may be explainable by simpler optimization processes.
	% [git]
	The code implementation for our experiments is available at:
	\href{https://github.com/GiilDe/ft-vs-icl}{https://github.com/GiilDe/ft-vs-icl}
\end{abstract}

\section{Introduction}
In recent studies \cite{pmlr-v202-von-oswald23a,dai2023gpt}, attempts have been made to establish a link between in-context learning (ICL) and the fine-tuning process using gradient descent (GD) in transformer models. One common conclusion that researchers frequently derive from these studies is that "ICL essentially enacts gradient descent." Nevertheless, it's worth noting that much of the analysis in these investigations focused on models with linear attention, which can be considered as a simplified setting.

In their study, \cite{dai2023gpt} expand their findings from linear attention to conventional attention mechanisms, relying on empirical evidence to support their assertions.
Their experiments convincingly demonstrate that a model fine-tuned through gradient descent steps and a model prompted with in-context examples appear to execute similar functions. In simpler terms, they exhibit analogous behaviors when processing inputs. Furthermore, they observe a substantial overlap in the internal behaviors of these two models.
Yet, it remains unclear how a transformer's forward pass calculates its backward pass, even if it's done in a clever way involving a smaller transformer embedded within the weights of the original transformer, as suggested by some. Even if this is theoretically achievable, it's a complex concept to put into practice. In contrast, when it comes to linear models, the gradient step has a straightforward mathematical formula, making it much simpler to handle.

In \cite{pmlr-v202-von-oswald23a}, they put forth a less ambitious idea. They suggest that the model essentially carries out a kind of gradient descent on a simple linear model applied to the original deep representations calculated during the forward pass.
Using linear models with detailed feature representations is known to be highly effective in accomplishing various tasks.


In this study, our goal is to demonstrate that this principle holds true in this context as well. This can support our argument that extensive fine-tuning is not required, and a basic linear model that uses the original hidden states as inputs suffices.
To do so, we perform three experiments:
\begin{itemize}
    \item \textbf{Linearization}: We replicated the similarities results of \cite{dai2023gpt} and compared them with a simplified version the function we optimize by linearization.
    \item \textbf{Gilad's experiment}: \textbf{TODO!!}
    \item \textbf{Labels Switching}: To further confirm the results of \cite{dai2023gpt}, motivated by textbf{TODO: add relevant labels experiments papers}, we evaluated the change in similarity when providing false labels both in FT and ICL.
\end{itemize}


\section{Background and Preliminaries}
\input{background.tex}


\section{Experiments}
\subsection{Evaluation Datasets}

We evaluated our experiments on six datasets. \textbf{SST2} \cite{socher-etal-2013-recursive} \textbf{SST5} \cite{socher-etal-2013-recursive}, \textbf{MR} \cite{10.3115/1219840.1219855} and \textbf{Subj} \cite{10.3115/1218955.1218990} are four datasets for sentiment classification; \textbf{AGNews} \cite{NIPS2015_250cf8b5} is a topic classification dataset; and \textbf{CB} \cite{Marneffe2019TheCI} is used
for natural language inference.

\subsection{Evaluation Metrics}

In the following sections we describe the evaluation metrics used to compare the behavior of ICL and finetuning.
To ensure an optimal comparison, we have adopted the identical metrics as introduced in \cite{dai2023gpt}:
We design three metrics to measure the similarity between ICL and finetuning at three different levels: the prediction level, the representation level, and the attention behavior level. 

\paragraph{Prediction Recall}

From the perspective of model prediction, models with similar behavior should have aligned predictions.
We measure the recall of correct ICL predictions to correct finetuning predictions.
Given a set of test examples, we count the subsets of examples correctly predicted by each model: $C_{\text{ZSL}}, C_{\text{ICL}}, C_{\text{FT}}$.
To compare the update each method induces to the model's prediction we subtract correct predictions made in the ZSL setting.
Finally we compute the \textbf{Rec2FTP} score as: $\frac{ \sizeof{ \left( C_{\text{ICL}} \cap C_{\text{FT}} \right) \setminus C_{\text{ZSL}} } }{ \sizeof{ C_{\text{FT}} \setminus C_{\text{ZSL}} } }$ .
A higher Rec2FTP score suggests that ICL covers more correct behavior of finetuning from the perspective of the model prediction.

%This measure is agnostic to the inner workings of the attention mechanism.

\paragraph{Attention Output Direction}
In the context of an attention layer's hidden state representation space within a model, we analyze the modifications made to the attention output representation (\textbf{SimAOU}).

For a given query example, let $h^{(l)}_X$ represent the normalized output representation of the last token at the $l$-th attention layer within setting $X$. The alterations introduced by ICL and finetuning in comparison to ZSL are denoted as $h^{(l)}_{ICL} - h^{(l)}_{ZSL}$ and $h^{(l)}{FT} - h^{(l)}_{ZSL}$, respectively. We calculate the cosine similarity between these two modifications to obtain SimAOU ($\Delta FT$) at the $l$-th layer. A higher SimAOU ($\Delta FT$) indicates that ICL is more inclined to adjust the attention output in the same direction as finetuning.
For the sake of comparison, we also compute a baseline metric known as SimAOU (Random $\Delta$), which measures the similarity between ICL updates and updates generated randomly.

\paragraph{Attention Map Similarity}
We use SimAM to measure the similarity between attention maps and query tokens for ICL and finetuning.
For a query example, let $m^{(l,h)}_X$ represent the attention weights before softmax in the $h$-th head of the $l$-th layer for setting $X$. In ICL, we focus solely on query token attention weights, excluding demonstration tokens. Initially, before finetuning, we compute the cosine similarity between $m^{(l,h)}_{ICL}$ and $m^{(l,h)}_{ZSL}$, averaging it across attention heads to obtain SimAM (Before Finetuning) for each layer.
Similarly, after finetuning, we calculate the cosine similarity between $m^{(l,h)}_{ICL}$ and $m^{(l,h)}_{FT}$ to obtain SimAM (After FT). A higher SimAM (After FT) relative to SimAM (Before FT) indicates that ICL's attention behavior aligns more with a finetuned model than a non-finetuned one.


\subsection{Prediction Alignment}
In this section we focus on the perspective of model predictions, regarding both ICL and finetuning as black-box updates to the zero-shot setting prediction.
While this analysis provides less insight into the inner workings of ICL, prediction alignment is easily interpretable and seems necessary for downstream applications of such comparisons.

We revisit the results of \cite{dai2023gpt}, and note the discrepancy between the accuracy of the finetuned model and the ICL setting.
Our evaluation shows an average relative difference of $19.38\%$ between ICL and finetuning accuracy with respect to the original model.
To better understand .... we measure .... reference fig

Specifically \cite{dai2023gpt} find that ICL achieves high \textbf{Rec2FTP} scores, which means it covers most of the correct predictions of finetuning.
%relative diff in acc by task (1_3B)[0.2666666666666666, 0.13994910941475827, 0.24279210925644915, 0.16804407713498629, 0.30021598272138245, 0.045333333333333406]$%
We argue that this metric is insufficient because (1) it does not account for the accuracy discrepancy, (2) it does not reflect incorrect prediction updates which may reflect meaningful behavioral changes in the model.


To address these fallacies we propose a measurement of \textbf{relative prediction alignment} (RPA) to quantify the similarity between both prediction updates.
Given a validation set, we denote the subset of example whose prediction is changed with regards to the zero-shot setting by ICL or FT by $D_\text{ICL}$ and $D_\text{FT}$ respectively.
We define the RPA of these updates by: $\frac{\sizeof{D_\text{ICL} \cap D_\text{FT}}}{\sizeof{D_\text{ICL}} + \sizeof{D_\text{FT}} - \sizeof{D_\text{ICL} \cap D_\text{FT}}}$.     


\input{}

\section{Discussion}
\subsection{Differences Between Theoretical Analysis and Proposed Methods}
Most works connecting ICL with gradient based optimization are motivated by theoretical intuition, or even include rigorous analysis for simplified settings \cite{pmlr-v202-von-oswald23a, akyürek2023learning} (see section~\ref{sec:related}).
Our work aims to demonstrate such connections empirically, guided by intuition provided in section \ref{sec:icl_dual_1}.  
It is important to note the differences between this analysis and practical settings: (1) It assumes linear attention is used (2) The analysis applies to the update of a single layer
(3) Starting from the ICL dual view, we get a partial-differential equation for the underlying loss function whose true value seems intractable.
While the empirical results provide direction for future research, we believe further analysis is need in order to gain more insight from such comparisons. 

\subsection{Limitations and Future Directions}
The method and results shown in section \ref{sec:layer_causality} are inconclusive.
We address its limitation using quantitative analysis, which suggests that further modifications may yield better similarity with ICL.
We leave such work to future research.  

\section{Related Works} \label{sec:related}
A series of recent works explores the similarities between ICL and gradient descent based optimization. 
\cite{akyürek2023learning} show that Transformer-based in-context learners can implement standard optimization algorithms on linear models implicitly.
\cite{pmlr-v202-von-oswald23a} provide a construction that for a linear attention-only Transformers models that implicitly perform gradient descent like procedure.
\cite{irie22dual} rewrite the dual form of a linear perceptron in terms of query-key-value attention, and use it to analyze how a trained model is affected by its training samples.

Different from these works, we base our study on \cite{dai2023gpt} which study large GPT transformers on structured language classification tasks.
We study how different aspects of ICL can affect the results of the comparison made in \cite{dai2023gpt}.
On the prediction level, we introduce the RPA metric for evaluation of prediction level alignment.
Furthermore, while \cite{dai2023gpt} compare standard GD based finetuning, we test a novel layer causality aware finetuning process.

\section{Conclusion}
Inspired by recent works, we attempted to further explore the relationship between in-context learning and gradient descent based finetuning in practical settings.
We revisit existing work and show that ICL and FT predictions are misaligned and propose a better measure quantify this difference.
Motivated by inherit differences in expressive power between ICL and FT, we explore whether a simpler variant of GD may show similar results - with insufficient results.
Finally, we address a fundamental difference in information flow between the methods, and suggest a novel FT method that respects layer causality.
Our results show potential for a more plausible explanation of ICL, and may suggest exciting possible practical applications for embedding context into a model's weights.  
\section{Acknowledgements}
We would like to express appreciation to Guy Dar who supervised this project, for his help with formulating the research question and methodology, helpful insights and continuous feedback.

\bibliography{anthology,refs}
\bibliographystyle{acl_natbib}

\newpage
\appendix
% \input{future.tex}

\end{document}
