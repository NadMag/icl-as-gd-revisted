% \pdfoutput=1

\documentclass[11pt]{article}

% \usepackage[review]{ACL2023}
\usepackage{ACL2023}

\usepackage{times}
\usepackage{latexsym}
\usepackage[T1]{fontenc}
\usepackage[utf8]{inputenc}
\usepackage{microtype}
\usepackage{inconsolata}
\usepackage{hyperref}
% \RequirePackage{algorithm}
% \RequirePackage{algorithmic}

\usepackage{multirow}
\usepackage{amsmath}
\usepackage{capt-of}
\usepackage{tabularx}
% \usepackage[caption=false]{subfig}
\usepackage{subcaption}
\usepackage{epsfig}
\usepackage{amssymb}
\usepackage{amsfonts}
\usepackage{booktabs}
\usepackage{scalerel}
%\usepackage[dvipsnames]{xcolor}
\usepackage[inline]{enumitem}
\usepackage{listings}
\usepackage{varwidth}
\usepackage[export]{adjustbox}
\usepackage{tikz}
\usetikzlibrary{tikzmark}
% \usepackage{todonotes}
% \usepackage{cleveref}
\newcommand{\crefrangeconjunction}{--}
\usepackage{stmaryrd}
\usepackage{bbm}
\usepackage{wrapfig}
\usepackage{pifont}

\newcommand{\tabincell}[2]{\begin{tabular}{@{}#1@{}}#2\end{tabular}}
\newcommand{\tx}[1]{``\textit{#1}''}
\newcommand{\sptk}[1]{\texttt{[#1]}}
% \newcommand{\eqform}[1]{Equation~(\ref{#1})}

% \usepackage{algorithm}
% \usepackage[noend]{algpseudocode}

\definecolor{deepblue}{rgb}{0,0,0.5}
\definecolor{officeblue}{RGB}{0,102,204}
\definecolor{deepred}{rgb}{0.6,0,0}
\definecolor{deepgreen}{rgb}{0,0.5,0}
\definecolor{mybrickred}{RGB}{182,50,28}
\newcommand\mybox[2][]{\tikz[overlay]\node[inner sep=1pt, anchor=text, rectangle, rounded corners=0mm,#1] {#2};\phantom{#2}}
\definecolor{fillcolor}{RGB}{216,217,252}
\newcommand\bg[1]{\mybox[fill=blue!20]{#1}}
\newcommand\rg[1]{\mybox[fill=red!20]{#1}}
\newcommand\graybox[1]{\mybox[fill=gray!20]{#1}}

% %%%%%Algorithm Box Package%%%%%
% \renewcommand{\algorithmicrequire}{\textbf{Input:}}
% \algnewcommand\algorithmicrequireb{{\hspace{0.85cm}}}
% \algnewcommand\INPTDESCB{\item[\algorithmicrequireb]}
% \renewcommand{\algorithmicensure}{\textbf{Output:}}
% \algnewcommand\algorithmicfuncdesc{\textbf{Function:}}
% \algnewcommand\FUNCDESC{\item[\algorithmicfuncdesc]}
% \algnewcommand\algorithmicfuncdescb{{\hspace{1.48cm}}}
% \algnewcommand\FUNCDESCB{\item[\algorithmicfuncdescb]}
% \algnewcommand{\algorithmicgoto}{\textbf{goto}}
% \algnewcommand{\Goto}[1]{\algorithmicgoto~\ref{#1}}
% \newcommand*\Let[2]{\State {#1 $\gets$ #2}}
% \newcommand*\LineLet[2]{#1 $\gets$ #2}
% %\newcommand*\AddOne[1]{\State #1 $\gets$ #1 $+1$}
% \newcommand*\AddOne[1]{\State #1 $++$}
% \newcommand*\LineComment[1]{\Statex \(\triangleright\) #1}
% \newcommand*\LineFor[2]{\State {\algorithmicfor~#1~\algorithmicdo~~~~#2}}
% \newcommand*\LineIf[2]{\State {\algorithmicif~#1~\algorithmicthen~~~~#2}}
% \newcommand*\AlgCommentInLine[1]{{\color{deepblue}{$\triangleright$ \textit{#1}}}}
% \newcommand*\AlgComment[1]{\State{\AlgCommentInLine{#1}}}
% %%%%%Algorithm Box Package%%%%%

\newcommand*\AlgCommentInLine[1]{{\color{deepblue}{$\triangleright$ \textit{#1}}}}

%%%%% NEW MATH DEFINITIONS %%%%%

\usepackage{amsmath,amsfonts,bm}

\newcommand\sizeof[1]{\left|#1\right|}

% Mark sections of captions for referring to divisions of figures
\newcommand{\figleft}{{\em (Left)}}
\newcommand{\figcenter}{{\em (Center)}}
\newcommand{\figright}{{\em (Right)}}
\newcommand{\figtop}{{\em (Top)}}
\newcommand{\figbottom}{{\em (Bottom)}}
\newcommand{\captiona}{{\em (a)}}
\newcommand{\captionb}{{\em (b)}}
\newcommand{\captionc}{{\em (c)}}
\newcommand{\captiond}{{\em (d)}}

% Highlight a newly defined term
\newcommand{\newterm}[1]{{\bf #1}}


% Figure reference, lower-case.
\def\figref#1{figure~\ref{#1}}
% Figure reference, capital. For start of sentence
\def\Figref#1{Figure~\ref{#1}}
\def\twofigref#1#2{figures \ref{#1} and \ref{#2}}
\def\quadfigref#1#2#3#4{figures \ref{#1}, \ref{#2}, \ref{#3} and \ref{#4}}
% Section reference, lower-case.
\def\secref#1{section~\ref{#1}}
% Section reference, capital.
\def\Secref#1{Section~\ref{#1}}
% Reference to two sections.
\def\twosecrefs#1#2{sections \ref{#1} and \ref{#2}}
% Reference to three sections.
\def\secrefs#1#2#3{sections \ref{#1}, \ref{#2} and \ref{#3}}
% Reference to an equation, lower-case.
\def\eqref#1{equation~\ref{#1}}
% Reference to an equation, upper case
\def\Eqref#1{Equation~\ref{#1}}
% A raw reference to an equation---avoid using if possible
\def\plaineqref#1{\ref{#1}}
% Reference to a chapter, lower-case.
\def\chapref#1{chapter~\ref{#1}}
% Reference to an equation, upper case.
\def\Chapref#1{Chapter~\ref{#1}}
% Reference to a range of chapters
\def\rangechapref#1#2{chapters\ref{#1}--\ref{#2}}
% Reference to an algorithm, lower-case.
\def\algref#1{algorithm~\ref{#1}}
% Reference to an algorithm, upper case.
\def\Algref#1{Algorithm~\ref{#1}}
\def\twoalgref#1#2{algorithms \ref{#1} and \ref{#2}}
\def\Twoalgref#1#2{Algorithms \ref{#1} and \ref{#2}}
% Reference to a part, lower case
\def\partref#1{part~\ref{#1}}
% Reference to a part, upper case
\def\Partref#1{Part~\ref{#1}}
\def\twopartref#1#2{parts \ref{#1} and \ref{#2}}

\def\ceil#1{\lceil #1 \rceil}
\def\floor#1{\lfloor #1 \rfloor}
\def\1{\bm{1}}
\newcommand{\train}{\mathcal{D}}
\newcommand{\valid}{\mathcal{D_{\mathrm{valid}}}}
\newcommand{\test}{\mathcal{D_{\mathrm{test}}}}

\def\eps{{\epsilon}}


% Random variables
\def\reta{{\textnormal{$\eta$}}}
\def\ra{{\textnormal{a}}}
\def\rb{{\textnormal{b}}}
\def\rc{{\textnormal{c}}}
\def\rd{{\textnormal{d}}}
\def\re{{\textnormal{e}}}
\def\rf{{\textnormal{f}}}
\def\rg{{\textnormal{g}}}
\def\rh{{\textnormal{h}}}
\def\ri{{\textnormal{i}}}
\def\rj{{\textnormal{j}}}
\def\rk{{\textnormal{k}}}
\def\rl{{\textnormal{l}}}
% rm is already a command, just don't name any random variables m
\def\rn{{\textnormal{n}}}
\def\ro{{\textnormal{o}}}
\def\rp{{\textnormal{p}}}
\def\rq{{\textnormal{q}}}
\def\rr{{\textnormal{r}}}
\def\rs{{\textnormal{s}}}
\def\rt{{\textnormal{t}}}
\def\ru{{\textnormal{u}}}
\def\rv{{\textnormal{v}}}
\def\rw{{\textnormal{w}}}
\def\rx{{\textnormal{x}}}
\def\ry{{\textnormal{y}}}
\def\rz{{\textnormal{z}}}

% Random vectors
\def\rvepsilon{{\mathbf{\epsilon}}}
\def\rvtheta{{\mathbf{\theta}}}
\def\rva{{\mathbf{a}}}
\def\rvb{{\mathbf{b}}}
\def\rvc{{\mathbf{c}}}
\def\rvd{{\mathbf{d}}}
\def\rve{{\mathbf{e}}}
\def\rvf{{\mathbf{f}}}
\def\rvg{{\mathbf{g}}}
\def\rvh{{\mathbf{h}}}
\def\rvu{{\mathbf{i}}}
\def\rvj{{\mathbf{j}}}
\def\rvk{{\mathbf{k}}}
\def\rvl{{\mathbf{l}}}
\def\rvm{{\mathbf{m}}}
\def\rvn{{\mathbf{n}}}
\def\rvo{{\mathbf{o}}}
\def\rvp{{\mathbf{p}}}
\def\rvq{{\mathbf{q}}}
\def\rvr{{\mathbf{r}}}
\def\rvs{{\mathbf{s}}}
\def\rvt{{\mathbf{t}}}
\def\rvu{{\mathbf{u}}}
\def\rvv{{\mathbf{v}}}
\def\rvw{{\mathbf{w}}}
\def\rvx{{\mathbf{x}}}
\def\rvy{{\mathbf{y}}}
\def\rvz{{\mathbf{z}}}

% Elements of random vectors
\def\erva{{\textnormal{a}}}
\def\ervb{{\textnormal{b}}}
\def\ervc{{\textnormal{c}}}
\def\ervd{{\textnormal{d}}}
\def\erve{{\textnormal{e}}}
\def\ervf{{\textnormal{f}}}
\def\ervg{{\textnormal{g}}}
\def\ervh{{\textnormal{h}}}
\def\ervi{{\textnormal{i}}}
\def\ervj{{\textnormal{j}}}
\def\ervk{{\textnormal{k}}}
\def\ervl{{\textnormal{l}}}
\def\ervm{{\textnormal{m}}}
\def\ervn{{\textnormal{n}}}
\def\ervo{{\textnormal{o}}}
\def\ervp{{\textnormal{p}}}
\def\ervq{{\textnormal{q}}}
\def\ervr{{\textnormal{r}}}
\def\ervs{{\textnormal{s}}}
\def\ervt{{\textnormal{t}}}
\def\ervu{{\textnormal{u}}}
\def\ervv{{\textnormal{v}}}
\def\ervw{{\textnormal{w}}}
\def\ervx{{\textnormal{x}}}
\def\ervy{{\textnormal{y}}}
\def\ervz{{\textnormal{z}}}

% Random matrices
\def\rmA{{\mathbf{A}}}
\def\rmB{{\mathbf{B}}}
\def\rmC{{\mathbf{C}}}
\def\rmD{{\mathbf{D}}}
\def\rmE{{\mathbf{E}}}
\def\rmF{{\mathbf{F}}}
\def\rmG{{\mathbf{G}}}
\def\rmH{{\mathbf{H}}}
\def\rmI{{\mathbf{I}}}
\def\rmJ{{\mathbf{J}}}
\def\rmK{{\mathbf{K}}}
\def\rmL{{\mathbf{L}}}
\def\rmM{{\mathbf{M}}}
\def\rmN{{\mathbf{N}}}
\def\rmO{{\mathbf{O}}}
\def\rmP{{\mathbf{P}}}
\def\rmQ{{\mathbf{Q}}}
\def\rmR{{\mathbf{R}}}
\def\rmS{{\mathbf{S}}}
\def\rmT{{\mathbf{T}}}
\def\rmU{{\mathbf{U}}}
\def\rmV{{\mathbf{V}}}
\def\rmW{{\mathbf{W}}}
\def\rmX{{\mathbf{X}}}
\def\rmY{{\mathbf{Y}}}
\def\rmZ{{\mathbf{Z}}}

% Elements of random matrices
\def\ermA{{\textnormal{A}}}
\def\ermB{{\textnormal{B}}}
\def\ermC{{\textnormal{C}}}
\def\ermD{{\textnormal{D}}}
\def\ermE{{\textnormal{E}}}
\def\ermF{{\textnormal{F}}}
\def\ermG{{\textnormal{G}}}
\def\ermH{{\textnormal{H}}}
\def\ermI{{\textnormal{I}}}
\def\ermJ{{\textnormal{J}}}
\def\ermK{{\textnormal{K}}}
\def\ermL{{\textnormal{L}}}
\def\ermM{{\textnormal{M}}}
\def\ermN{{\textnormal{N}}}
\def\ermO{{\textnormal{O}}}
\def\ermP{{\textnormal{P}}}
\def\ermQ{{\textnormal{Q}}}
\def\ermR{{\textnormal{R}}}
\def\ermS{{\textnormal{S}}}
\def\ermT{{\textnormal{T}}}
\def\ermU{{\textnormal{U}}}
\def\ermV{{\textnormal{V}}}
\def\ermW{{\textnormal{W}}}
\def\ermX{{\textnormal{X}}}
\def\ermY{{\textnormal{Y}}}
\def\ermZ{{\textnormal{Z}}}

% Vectors
\def\vzero{{\bm{0}}}
\def\vone{{\bm{1}}}
\def\vmu{{\bm{\mu}}}
\def\vtheta{{\bm{\theta}}}
\def\va{{\bm{a}}}
\def\vb{{\bm{b}}}
\def\vc{{\bm{c}}}
\def\vd{{\bm{d}}}
\def\ve{{\bm{e}}}
\def\vf{{\bm{f}}}
\def\vg{{\bm{g}}}
\def\vh{{\bm{h}}}
\def\vi{{\bm{i}}}
\def\vj{{\bm{j}}}
\def\vk{{\bm{k}}}
\def\vl{{\bm{l}}}
\def\vm{{\bm{m}}}
\def\vn{{\bm{n}}}
\def\vo{{\bm{o}}}
\def\vp{{\bm{p}}}
\def\vq{{\bm{q}}}
\def\vr{{\bm{r}}}
\def\vs{{\bm{s}}}
\def\vt{{\bm{t}}}
\def\vu{{\bm{u}}}
\def\vv{{\bm{v}}}
\def\vw{{\bm{w}}}
\def\vx{{\bm{x}}}
\def\vy{{\bm{y}}}
\def\vz{{\bm{z}}}

% Elements of vectors
\def\evalpha{{\alpha}}
\def\evbeta{{\beta}}
\def\evepsilon{{\epsilon}}
\def\evlambda{{\lambda}}
\def\evomega{{\omega}}
\def\evmu{{\mu}}
\def\evpsi{{\psi}}
\def\evsigma{{\sigma}}
\def\evtheta{{\theta}}
\def\eva{{a}}
\def\evb{{b}}
\def\evc{{c}}
\def\evd{{d}}
\def\eve{{e}}
\def\evf{{f}}
\def\evg{{g}}
\def\evh{{h}}
\def\evi{{i}}
\def\evj{{j}}
\def\evk{{k}}
\def\evl{{l}}
\def\evm{{m}}
\def\evn{{n}}
\def\evo{{o}}
\def\evp{{p}}
\def\evq{{q}}
\def\evr{{r}}
\def\evs{{s}}
\def\evt{{t}}
\def\evu{{u}}
\def\evv{{v}}
\def\evw{{w}}
\def\evx{{x}}
\def\evy{{y}}
\def\evz{{z}}

% Matrix
\def\mA{{\bm{A}}}
\def\mB{{\bm{B}}}
\def\mC{{\bm{C}}}
\def\mD{{\bm{D}}}
\def\mE{{\bm{E}}}
\def\mF{{\bm{F}}}
\def\mG{{\bm{G}}}
\def\mH{{\bm{H}}}
\def\mI{{\bm{I}}}
\def\mJ{{\bm{J}}}
\def\mK{{\bm{K}}}
\def\mL{{\bm{L}}}
\def\mM{{\bm{M}}}
\def\mN{{\bm{N}}}
\def\mO{{\bm{O}}}
\def\mP{{\bm{P}}}
\def\mQ{{\bm{Q}}}
\def\mR{{\bm{R}}}
\def\mS{{\bm{S}}}
\def\mT{{\bm{T}}}
\def\mU{{\bm{U}}}
\def\mV{{\bm{V}}}
\def\mW{{\bm{W}}}
\def\mX{{\bm{X}}}
\def\mY{{\bm{Y}}}
\def\mZ{{\bm{Z}}}
\def\mBeta{{\bm{\beta}}}
\def\mPhi{{\bm{\Phi}}}
\def\mLambda{{\bm{\Lambda}}}
\def\mSigma{{\bm{\Sigma}}}

% Tensor
\DeclareMathAlphabet{\mathsfit}{\encodingdefault}{\sfdefault}{m}{sl}
\SetMathAlphabet{\mathsfit}{bold}{\encodingdefault}{\sfdefault}{bx}{n}
\newcommand{\tens}[1]{\bm{\mathsfit{#1}}}
\def\tA{{\tens{A}}}
\def\tB{{\tens{B}}}
\def\tC{{\tens{C}}}
\def\tD{{\tens{D}}}
\def\tE{{\tens{E}}}
\def\tF{{\tens{F}}}
\def\tG{{\tens{G}}}
\def\tH{{\tens{H}}}
\def\tI{{\tens{I}}}
\def\tJ{{\tens{J}}}
\def\tK{{\tens{K}}}
\def\tL{{\tens{L}}}
\def\tM{{\tens{M}}}
\def\tN{{\tens{N}}}
\def\tO{{\tens{O}}}
\def\tP{{\tens{P}}}
\def\tQ{{\tens{Q}}}
\def\tR{{\tens{R}}}
\def\tS{{\tens{S}}}
\def\tT{{\tens{T}}}
\def\tU{{\tens{U}}}
\def\tV{{\tens{V}}}
\def\tW{{\tens{W}}}
\def\tX{{\tens{X}}}
\def\tY{{\tens{Y}}}
\def\tZ{{\tens{Z}}}


% Graph
\def\gA{{\mathcal{A}}}
\def\gB{{\mathcal{B}}}
\def\gC{{\mathcal{C}}}
\def\gD{{\mathcal{D}}}
\def\gE{{\mathcal{E}}}
\def\gF{{\mathcal{F}}}
\def\gG{{\mathcal{G}}}
\def\gH{{\mathcal{H}}}
\def\gI{{\mathcal{I}}}
\def\gJ{{\mathcal{J}}}
\def\gK{{\mathcal{K}}}
\def\gL{{\mathcal{L}}}
\def\gM{{\mathcal{M}}}
\def\gN{{\mathcal{N}}}
\def\gO{{\mathcal{O}}}
\def\gP{{\mathcal{P}}}
\def\gQ{{\mathcal{Q}}}
\def\gR{{\mathcal{R}}}
\def\gS{{\mathcal{S}}}
\def\gT{{\mathcal{T}}}
\def\gU{{\mathcal{U}}}
\def\gV{{\mathcal{V}}}
\def\gW{{\mathcal{W}}}
\def\gX{{\mathcal{X}}}
\def\gY{{\mathcal{Y}}}
\def\gZ{{\mathcal{Z}}}

% Sets
\def\sA{{\mathbb{A}}}
\def\sB{{\mathbb{B}}}
\def\sC{{\mathbb{C}}}
\def\sD{{\mathbb{D}}}
% Don't use a set called E, because this would be the same as our symbol
% for expectation.
\def\sF{{\mathbb{F}}}
\def\sG{{\mathbb{G}}}
\def\sH{{\mathbb{H}}}
\def\sI{{\mathbb{I}}}
\def\sJ{{\mathbb{J}}}
\def\sK{{\mathbb{K}}}
\def\sL{{\mathbb{L}}}
\def\sM{{\mathbb{M}}}
\def\sN{{\mathbb{N}}}
\def\sO{{\mathbb{O}}}
\def\sP{{\mathbb{P}}}
\def\sQ{{\mathbb{Q}}}
\def\sR{{\mathbb{R}}}
\def\sS{{\mathbb{S}}}
\def\sT{{\mathbb{T}}}
\def\sU{{\mathbb{U}}}
\def\sV{{\mathbb{V}}}
\def\sW{{\mathbb{W}}}
\def\sX{{\mathbb{X}}}
\def\sY{{\mathbb{Y}}}
\def\sZ{{\mathbb{Z}}}

% Entries of a matrix
\def\emLambda{{\Lambda}}
\def\emA{{A}}
\def\emB{{B}}
\def\emC{{C}}
\def\emD{{D}}
\def\emE{{E}}
\def\emF{{F}}
\def\emG{{G}}
\def\emH{{H}}
\def\emI{{I}}
\def\emJ{{J}}
\def\emK{{K}}
\def\emL{{L}}
\def\emM{{M}}
\def\emN{{N}}
\def\emO{{O}}
\def\emP{{P}}
\def\emQ{{Q}}
\def\emR{{R}}
\def\emS{{S}}
\def\emT{{T}}
\def\emU{{U}}
\def\emV{{V}}
\def\emW{{W}}
\def\emX{{X}}
\def\emY{{Y}}
\def\emZ{{Z}}
\def\emSigma{{\Sigma}}

% entries of a tensor
% Same font as tensor, without \bm wrapper
\newcommand{\etens}[1]{\mathsfit{#1}}
\def\etLambda{{\etens{\Lambda}}}
\def\etA{{\etens{A}}}
\def\etB{{\etens{B}}}
\def\etC{{\etens{C}}}
\def\etD{{\etens{D}}}
\def\etE{{\etens{E}}}
\def\etF{{\etens{F}}}
\def\etG{{\etens{G}}}
\def\etH{{\etens{H}}}
\def\etI{{\etens{I}}}
\def\etJ{{\etens{J}}}
\def\etK{{\etens{K}}}
\def\etL{{\etens{L}}}
\def\etM{{\etens{M}}}
\def\etN{{\etens{N}}}
\def\etO{{\etens{O}}}
\def\etP{{\etens{P}}}
\def\etQ{{\etens{Q}}}
\def\etR{{\etens{R}}}
\def\etS{{\etens{S}}}
\def\etT{{\etens{T}}}
\def\etU{{\etens{U}}}
\def\etV{{\etens{V}}}
\def\etW{{\etens{W}}}
\def\etX{{\etens{X}}}
\def\etY{{\etens{Y}}}
\def\etZ{{\etens{Z}}}

% The true underlying data generating distribution
\newcommand{\pdata}{p_{\rm{data}}}
% The empirical distribution defined by the training set
\newcommand{\ptrain}{\hat{p}_{\rm{data}}}
\newcommand{\Ptrain}{\hat{P}_{\rm{data}}}
% The model distribution
\newcommand{\pmodel}{p_{\rm{model}}}
\newcommand{\Pmodel}{P_{\rm{model}}}
\newcommand{\ptildemodel}{\tilde{p}_{\rm{model}}}
% Stochastic autoencoder distributions
\newcommand{\pencode}{p_{\rm{encoder}}}
\newcommand{\pdecode}{p_{\rm{decoder}}}
\newcommand{\precons}{p_{\rm{reconstruct}}}

\newcommand{\laplace}{\mathrm{Laplace}} % Laplace distribution

\newcommand{\E}{\mathbb{E}}
\newcommand{\Ls}{\mathcal{L}}
\newcommand{\R}{\mathbb{R}}
\newcommand{\emp}{\tilde{p}}
\newcommand{\lr}{\alpha}
\newcommand{\reg}{\lambda}
\newcommand{\rect}{\mathrm{rectifier}}
\newcommand{\softmax}{\mathrm{softmax}}
\newcommand{\sigmoid}{\sigma}
\newcommand{\softplus}{\zeta}
\newcommand{\KL}{D_{\mathrm{KL}}}
\newcommand{\Var}{\mathrm{Var}}
\newcommand{\standarderror}{\mathrm{SE}}
\newcommand{\Cov}{\mathrm{Cov}}
\newcommand{\diag}{\mathrm{diag}}
% Wolfram Mathworld says $L^2$ is for function spaces and $\ell^2$ is for vectors
% But then they seem to use $L^2$ for vectors throughout the site, and so does
% wikipedia.
\newcommand{\normlzero}{L^0}
\newcommand{\normlone}{L^1}
\newcommand{\normltwo}{L^2}
\newcommand{\normlp}{L^p}
\newcommand{\normmax}{L^\infty}

\newcommand{\parents}{Pa} % See usage in notation.tex. Chosen to match Daphne's book.

\DeclareMathOperator*{\argmax}{arg\,max}
\DeclareMathOperator*{\argmin}{arg\,min}

\DeclareMathOperator{\sign}{sign}
\DeclareMathOperator{\Tr}{Tr}
\let\ab\allowbreak


% \newcommand{\cmark}{\ding{51}\xspace}%
\newcommand{\cmarkg}{\textcolor{lightgray}{\ding{51}}\xspace}%
% \newcommand{\xmark}{\ding{55}\xspace}%
\newcommand{\xmarkg}{\textcolor{lightgray}{\ding{55}}\xspace}%

\newcommand\our{\textsc{structured prompting}}
\newcommand{\tblidx}[1]{{\scriptsize \texttt{[#1]}}}
\newtheorem{theorem}{Property}

\usepackage{pifont}% http://ctan.org/pkg/pifont
\newcommand{\cmark}{{\color{blue}\ding{51}}}%
\newcommand{\xmark}{{\color{black}\ding{55}}}%


\title{Rethinking In-Context Learning in Large Language Models as Gradient Descent}


\newcommand*\samethanks[1][\value{footnote}]{\footnotemark[#1]}

\author{\\
Tomer Bar Natan,~Gilad Deutch,~Nadav Magar\\
~~~~Supervised by Guy Dar\\
~~~~~Tel-Aviv University}

\date{}

\begin{document}

\maketitle

\begin{abstract}
	In-context learning (ICL) has shown impressive results in few-shot learning tasks, yet its underlying mechanism is still not fully understood.
	Recent works suggest that ICL could be thought of as a gradient descent (GD) based meta-optimization process.
	While promising, these results mainly focus on simplified settings of ICL and provide only a preliminary evaluation of the similarities between the two methods. 
	In this work, we revisit the comparison between ICL and GD-based finetuning and study what properties of ICL an equivalent process must follow. 
	% Prediction level
	We begin at the model prediction level and reexamine how ICL and finetuning's prediction updates differ.
	%Motivated by these results we propose a novel evaluation metric which we term relative prediction alignment (RPA).
	% Layer Causality
	Next, we address the differences in layer causality between ICL and standard finetuning.
	To study how this dissimilarity affects the model's behavior we propose a causally aware finetuning process and compare it with previous results. 
	%We find that the causally aware process competes with standard finetuning across several comparison metrics.
	% Linearization
	% Lastly we show that finetuning using a linear approximation of the model achieves comparable results to standard finetuning,
	% suggesting that the underlying behavior may be explainable by simpler optimization processes.
	% [git]
	The code implementation for our experiments is available at:
	\href{https://github.com/GiilDe/ft-vs-icl}{https://github.com/GiilDe/ft-vs-icl}
\end{abstract}

\section{Introduction}
In recent studies \cite{pmlr-v202-von-oswald23a,dai2023gpt}, attempts have been made to establish a link between in-context learning (ICL) and the fine-tuning process using gradient descent (GD) in transformer models. One common conclusion that researchers frequently derive from these studies is that "ICL essentially enacts gradient descent." Nevertheless, it's worth noting that much of the analysis in these investigations focused on models with linear attention, which can be considered as a simplified setting.

In their study, \cite{dai2023gpt} expand their findings from linear attention to conventional attention mechanisms, relying on empirical evidence to support their assertions.
Their experiments convincingly demonstrate that a model fine-tuned through gradient descent steps and a model prompted with in-context examples appear to execute similar functions. In simpler terms, they exhibit analogous behaviors when processing inputs. Furthermore, they observe a substantial overlap in the internal behaviors of these two models.
Yet, it remains unclear how a transformer's forward pass calculates its backward pass, even if it's done in a clever way involving a smaller transformer embedded within the weights of the original transformer, as suggested by some. Even if this is theoretically achievable, it's a complex concept to put into practice. In contrast, when it comes to linear models, the gradient step has a straightforward mathematical formula, making it much simpler to handle.

In \cite{pmlr-v202-von-oswald23a}, they put forth a less ambitious idea. They suggest that the model essentially carries out a kind of gradient descent on a simple linear model applied to the original deep representations calculated during the forward pass.
Using linear models with detailed feature representations is known to be highly effective in accomplishing various tasks.


In this study, our goal is to demonstrate that this principle holds true in this context as well. This can support our argument that extensive fine-tuning is not required, and a basic linear model that uses the original hidden states as inputs suffices.
To do so, we perform three experiments:
\begin{itemize}
    \item \textbf{Linearization}: We replicated the similarities results of \cite{dai2023gpt} and compared them with a simplified version the function we optimize by linearization.
    \item \textbf{Gilad's experiment}: \textbf{TODO!!}
    \item \textbf{Labels Switching}: To further confirm the results of \cite{dai2023gpt}, motivated by textbf{TODO: add relevant labels experiments papers}, we evaluated the change in similarity when providing false labels both in FT and ICL.
\end{itemize}


\section{Background and Preliminaries}
\subsection{Dual Form Between Attention and Linear Layers Optimized by Gradient Descent}

The view of language models as meta-optimizers originates from the presentation of the dual and primal forms of the perceptron \cite{Aizerman2019TheoreticalFO}.
This notion was later expressed in terms key/value/query-attention operation by by \cite{irie22dual,dai2023gpt,pmlr-v202-von-oswald23a} which apply it apply it in the modern context of deep neural networks.
They show that linear layers optimized by gradient descent have a dual representation as linear attention.

Let $W \in \mathbb{R}^{d_{\text{out}} \times d_{\text{in}}}$ be the weight matrix of a linear layer initialized at $W_0$, and let $\mathbf{x}, \mathbf{x}_1, \dots, \mathbf{x}_n  \in \mathbb{R}^{d_{\text{in}}}$ be the input and training examples representation respectively.
One step of gradient descent on the loss function $\mathcal{L}$ with learning rate $\eta$ yields the weight update $\Delta W$.
This update can be written as the outer products of the training examples $\mathbf{x}_1, \dots, \mathbf{x}_n$ and the gradients of their corresponding outputs $\mathbf{e}_i = -\eta \nabla_{W_0 x_i}\mathcal{L}$
\begin{equation}
    \Delta W = \sum_i \mathbf{e}_i \otimes \mathbf{x}^{\prime T}_i.
    \label{equ:dual_comp_2}
\end{equation}

Thus the computation of the optimized linear layer can be formulated as 

\begin{equation}
    \begin{aligned}
        \mathcal{F}(\mathbf{x}) = & \left( W_{0} + \Delta W \right) \mathbf{x} \\
        = & W_{0} \mathbf{x} + \Delta W \mathbf{x} \\
        = & W_{0} \mathbf{x} + \sum_i \left( \mathbf{e}_i \otimes \mathbf{x}_i\right) \mathbf{x} \\
        = & W_{0} \mathbf{x} + \sum_i \mathbf{e}_i \left( \mathbf{x}^{T}_i \mathbf{x} \right) \\
        = & W_{0} \mathbf{x} + \operatorname{LinearAttn} \left( E, X, \mathbf{x} \right), 
    \end{aligned}
    \label{equ:sgd_attn_dual}
\end{equation}
where $\operatorname{LinearAttn}(V, K, \mathbf{q})$ denotes the linear attention operation.
From the perspective of attention we regard training examples $X$ as keys, their corresponding gradients as values, and the current input $\mathbf{x}$ as the query.


\subsection{Understanding Transformer Attention as Meta-Optimization}
\label{sec:icl_dual}
In this section we explain the simplified mathematical view of in-context learning as a process of meta-optimization presented in \cite{dai2023gpt}.
For the purpose of analysis, it is useful to view the change to the output induced by attention to the demonstration tokens as equivalent parameter update $\Delta W_{\text{ICL}}$ that take effect on the original attention parameters.

Let $\mathbf{x} \in \mathbb{R}^{d}$ be the input representation of a query token $t$, and $\mathbf{q} = W_{Q} \mathbf{x} \in \mathbb{R}^{d^{\prime}}$ be the attention query vector. 
We use the relaxed linear attention model, whereby the softmax operation and the scaling factor are omitted:
\begin{equation}
    \begin{aligned}
    \mathcal{F}_{\text{ICL}}(\mathbf{q}) & = \operatorname{LinearAttn}(V, K, \mathbf{q}) \\
    & = W_{V} [X^{\prime}; X] \left( W_{K} [X^{\prime}; X] \right)^T \mathbf{q} \\
    \end{aligned}
    \label{equ:icl_attn}
\end{equation}
where $W_{Q}, W_{K}, W_{V} \in \mathbb{R}^{d^{\prime} \times d}$ are the projection matrices for computing the attention queries, keys, and values, respectively; 
$X$ denotes the input representations of query tokens before $t$; 
$X^{\prime}$ denotes the input representations of the demonstration tokens; 
and $[X^{\prime}; X]$ denotes the matrix concatenation. 


They define $W_{\text{ZSL}} = W_{V} X \left( W_{K} X \right)^T$ as the initial parameters of a linear layer that is updated by attention to in-context demonstrations.
To see this, note that $W_{\text{ZSL}}$ is the attention result in the zero-shot learning setting where no demonstrations are given (Equation \ref{equ:icl_attn}). 
Following the reverse direction of Equation (\ref{equ:sgd_attn_dual}), you arrive at the dual form of the Transformer attention: 
\begin{equation}
    \begin{aligned}
        \mathcal{F}_{\text{ICL}}(\mathbf{q})
        = & W_{\text{ZSL}} \mathbf{q} + \operatorname{LinearAttn} \left( W_{V} X^{\prime}, W_{K} X^{\prime}, \mathbf{q} \right) \\
        = & W_{\text{ZSL}} \mathbf{q} + \sum_i W_{V} \textbf{x}^{\prime}_i \left( \left( W_{K} \textbf{x}^{\prime}_i \right)^T \mathbf{q} \right) \\
        = & W_{\text{ZSL}} \mathbf{q} + \sum_i \left( W_{V} \textbf{x}^{\prime}_i \otimes \left( W_{K} \textbf{x}^{\prime}_i \right) \right) \mathbf{q} \\
        = & W_{\text{ZSL}} \mathbf{q} + \Delta W_{\text{ICL}} \mathbf{q} \\
        = & \left( W_{\text{ZSL}} + \Delta W_{\text{ICL}} \right) \mathbf{q}. 
    \end{aligned}
    \label{equ:icl_opti_dual}
\end{equation}

By analogy with Equation(\ref{equ:sgd_attn_dual}), we can regard $W_{K} \textbf{x}^{\prime}_i$ as the training examples and $W_{V} X^{\prime}$ as their corresponding meta-gradients. 

\subsection{Linearization of a Model}
Resent works suggest a function we optimize can significantly simplify by a method called linearization \cite{10.5555/3454287.3455056, linearization23}.
Specifically, these works suggest that it is possible to approximate a pre-trained model in the vicinity of its initial parameters:
\begin{equation}
    f_{\theta_0 + \delta \theta}(x) \approx f_{\theta_0}(x) + \delta \theta^\textrm{T} \nabla_{\theta_0} f(x) := f_{\delta \theta}^{\textrm{lin}}(x; \theta_0)
\end{equation}
where $\theta_0$ represents the pre-trained model's parameters, $\delta \theta$ indicates the change in parameters during fine-tuning, and $x$ denotes an input sequence (fixed with respect to $\delta \theta$).
This is a linear model across deep representations, denoted as $\phi_i(x) = \nabla_{\theta_0} f(x)$. We don’t require taking gradients through the entire network, just the coefficients. We then perform the following gradient step:

\[
\theta \xleftarrow{} \theta - \eta \nabla\mathcal{L}(f^{\textrm{lin}} (x), y) \cdot \phi(x)
\]

In our work, we underscore the importance of exploring alternative approaches that enable the implementation of functions with reduced complexity when investigating ICL. This emphasis arises from the limitations imposed on the complexity of functions attainable through the forward pass, which must align with the network's depth. Therefore, we have explored a linearized variant of the model to address this constraint


\section{Experiments}
\subsection{Evaluation Metrics}

In the following sections we describe the evaluation metrics used to compare the behavior of ICL and finetuning.
To ensure an optimal comparison, we have adopted the identical metrics as introduced in \cite{dai2023gpt}:
We design three metrics to measure the similarity between ICL and finetuning at three different levels: the prediction level, the representation level, and the attention behavior level. 

\paragraph{Prediction Recall}

From the perspective of model prediction, models with similar behavior should have aligned predictions.
We measure the recall of correct ICL predictions to correct finetuning predictions.
Given a set of test examples, we count the subsets of examples correctly predicted by each model: $C_{\text{ZSL}}, C_{\text{ICL}}, C_{\text{FT}}$.
To compare the update each method induces to the model's prediction we subtract correct predictions made in the ZSL setting.
Finally we compute the \textbf{Rec2FTP} score as: $\frac{ \sizeof{ \left( C_{\text{ICL}} \cap C_{\text{FT}} \right) \setminus C_{\text{ZSL}} } }{ \sizeof{ C_{\text{FT}} \setminus C_{\text{ZSL}} } }$ .
A higher Rec2FTP score suggests that ICL covers more correct behavior of finetuning from the perspective of the model prediction.

%This measure is agnostic to the inner workings of the attention mechanism.

\paragraph{Attention Output Direction}
In the context of an attention layer's hidden state representation space within a model, we analyze the modifications made to the attention output representation (\textbf{SimAOU}).

For a given query example, let $h^{(l)}_X$ represent the normalized output representation of the last token at the $l$-th attention layer within setting $X$. The alterations introduced by ICL and finetuning in comparison to ZSL are denoted as $h^{(l)}_{ICL} - h^{(l)}_{ZSL}$ and $h^{(l)}{FT} - h^{(l)}_{ZSL}$, respectively. We calculate the cosine similarity between these two modifications to obtain SimAOU ($\Delta FT$) at the $l$-th layer. A higher SimAOU ($\Delta FT$) indicates that ICL is more inclined to adjust the attention output in the same direction as finetuning.
For the sake of comparison, we also compute a baseline metric known as SimAOU (Random $\Delta$), which measures the similarity between ICL updates and updates generated randomly.

\paragraph{Attention Map Similarity}
We use SimAM to measure the similarity between attention maps and query tokens for ICL and finetuning.
For a query example, let $m^{(l,h)}_X$ represent the attention weights before softmax in the $h$-th head of the $l$-th layer for setting $X$. In ICL, we focus solely on query token attention weights, excluding demonstration tokens. Initially, before finetuning, we compute the cosine similarity between $m^{(l,h)}_{ICL}$ and $m^{(l,h)}_{ZSL}$, averaging it across attention heads to obtain SimAM (Before Finetuning) for each layer.
Similarly, after finetuning, we calculate the cosine similarity between $m^{(l,h)}_{ICL}$ and $m^{(l,h)}_{FT}$ to obtain SimAM (After FT). A higher SimAM (After FT) relative to SimAM (Before FT) indicates that ICL's attention behavior aligns more with a finetuned model than a non-finetuned one.

\section{Discussion}
\subsection{Differences Between Theoretical Analysis and Proposed Methods}
Most works connecting ICL with gradient-based optimization are motivated by theoretical intuition or even include rigorous analysis for simplified settings \cite{pmlr-v202-von-oswald23a, akyürek2023learning} (see section~\ref{sec:related}).
Our work aims to demonstrate such connections empirically, guided by intuition provided in section \ref{sec:icl_dual_1}.  
It is important to note the differences between this analysis and practical settings: (1) It assumes linear attention is used (2) The analysis applies to the update of a single layer
(3) Starting from the ICL dual view, we get a partial-differential equation for the underlying loss function whose true value seems intractable.
While the empirical results provide direction for future research, we believe further analysis is needed in order to gain more insight from such comparisons. 

\subsection{Limitations and Future Directions}
The method and results shown in section \ref{sec:layer_causality} are inconclusive.
We address its limitation using quantitative analysis, which suggests that further modifications may yield better similarity with ICL.
We leave such work to future research.  

\section{Related Works} \label{sec:related}
A series of recent works explores the similarities between ICL and gradient descent-based optimization. 
\cite{akyürek2023learning} show that Transformer-based in-context learners can implement standard optimization algorithms on linear models implicitly.
\cite{pmlr-v202-von-oswald23a} provide a construction for linear attention-only Transformers models that implicitly perform gradient descent like procedure.
\cite{irie22dual} rewrite the dual form of a linear perceptron in terms of query-key-value attention, and use it to analyze how a trained model is affected by its training samples.

Different from these works, we base our study on \cite{dai2023gpt} which studies large GPT transformers on structured language classification tasks.
We study how different aspects of ICL can affect the results of the comparison made in \cite{dai2023gpt}.
On the prediction level, we introduce the RPA metric for the evaluation of prediction level alignment.
Furthermore, while \cite{dai2023gpt} compares standard GD-based finetuning, we test a novel layer causality-aware finetuning process.

\section{Conclusion}
Inspired by recent works, we attempted to further explore the relationship between in-context learning and gradient descent-based finetuning in practical settings.
We revisit existing work and show that ICL and FT predictions are misaligned and propose a better measure to quantify this difference.
% Motivated by inherited differences in expressive power between ICL and FT, we explore whether a simpler variant of GD may show similar results - with insufficient results.
Finally, we address a fundamental difference in information flow between the methods and suggest a novel FT method that respects layer causality.
Our results show potential for a more plausible explanation of ICL and may suggest exciting possible practical applications for embedding context into a model's weights.  
\section{Acknowledgements}
We would like to express appreciation to Guy Dar who supervised this project, for his help with formulating the research question and methodology, helpful insights and continuous feedback.

\bibliography{anthology,refs}
\bibliographystyle{acl_natbib}

\newpage
\appendix
% \input{future.tex}

\end{document}
